\documentclass[11pt]{article}

\usepackage{amsmath}
\usepackage{textcomp}

% Add other packages here %


% Put your group number and names in the author field %
\title{\bf Excercise 4\\ Implementing a centralized agent}
\author{Group \textnumero : Student 1, Student 2}


% N.B.: The report should not be longer than 3 pages %


\begin{document}
\maketitle

\section{Solution Representation}

\subsection{Variables}
The idea is to use the provided CSP and decompose a task into a pickup action and delivery action:

nextAction: 
an array of $N_A +N_V$ variables. 
The array contains one variable	for every existing action, and one variable for every existing vehicle:
$nextAction = [nextAction(a1), . . . , nextAction(a_{N_A}
), nextAction(v1), . . ., nextAction(v_{N_V}
)]$
One variable from the nextAction array can take as a value another action,
or the value NULL:
$nextAction(x) \in A \cup \{NULL\}$, where for every task, there is a pickup and delivery action:
$A = \{p(t), a(t) \colon t \text{ is a tast}\}$.
with the following semantics:
\begin{itemize}
	\item  if $nextAction(a_i) = a_j$ it means that some vehicle will do action $a_j$ immediately after $a_i$;
	\item  $if nextAction(v_k) = a_j$ it means that the vehicle $v_k$ does action $a_j$ (pickup) at first ;
	\item if $nextAction(a_i) = NULL$, the vehicle that did action $a_i$ (delivery) has nothing left to do;
	\item if $nextAction(vk) = NULL$, the vehicle vk does not have to any action.
\end{itemize}

One variable from the vehicle array can take as a value the code of the
vehicle that delivers the corresponding task:
$$ vehicle(a) \in V$$

\subsubsection{Implementation}
We implemented this a bit differently:
The datastructure is a map of lists of action.
Every list corresponds to the actions one vehicle has to perform.
The first element of the list of vehicle $v$ is $nextAction(v)$.
The successor of an action in the list is the result of nextAction.

\subsection{Constraints}
% Describe the constraints in your solution representation %
\begin{enumerate}
	\item $nextAction(a) \neq a$: the action performed after some action $a$ cannot be the same
	action;
	\item All tasks have to be delivered:
	the set of actions in $nextAction$ has to equal $A$ plus $N_V$ times NULL.
	\item CAPACITY TODO	
	\item $nextAction(v) = a \implies vehicle(a) = v$
	\item $nextAction(a) = b \impliedby vehicle(a) = vehicle(b)$
\end{enumerate}
\subsection{Objective function}
% Describe the function that you optimize %
\begin{itemize}
	\item $dist(a,b)$ is the shortest distance between the cities corresponding to the task of $a$ and $b$, i.e.
	if $a$ is a pickup, then the corresponding city is the pickup location of the task of $a$, otherwise the delivery city.
	\item $dist(a, NULL) = 0$, the vehicle stops after the last action.
	\item $dist(v, a)$  is the shortest distance between the home location of the vehicle
	$v$ and the city of $a$;
	\item $dist(v, NULL) = 0$;
	\item $cost(v)$ is the cost per kilometer of vehicle $v$.
\end{itemize}
The total cost of the company is defined as:
\begin{align*}
	C = \sum_{a \in A} dist(a, nextAction(a)) \cdot cost(vehicle(a)) + \sum_{v \in V} dist(v, nextAction(v)) \cdot cost(v)
\end{align*}
\section{Stochastic optimization}

\subsection{Initial solution}
% Describe how you generate the initial solution %

\subsection{Generating neighbours}
% Describe how you generate neighbors %

\subsection{Stochastic optimization algorithm}
% Describe your stochastic optimization algorithm %


\section{Results}

\subsection{Experiment 1: Model parameters}
% if your model has parameters, perform an experiment and analyze the results for different parameter values %

\subsubsection{Setting}
% Describe the settings of your experiment: topology, task configuration, number of tasks, number of vehicles, etc. %
% and the parameters you are analyzing %

\subsubsection{Observations}
% Describe the experimental results and the conclusions you inferred from these results %

\subsection{Experiment 2: Different configurations}
% Run simulations for different configurations of the environment (i.e. different tasks and number of vehicles) %

\subsubsection{Setting}
% Describe the settings of your experiment: topology, task configuration, number of tasks, number of vehicles, etc. %

\subsubsection{Observations}
% Describe the experimental results and the conclusions you inferred from these results %
% Reflect on the fairness of the optimal plans. Observe that optimality requires some vehicles to do more work than others. %
% How does the complexity of your algorithm depend on the number of vehicles and various sizes of the task set? %

\end{document}