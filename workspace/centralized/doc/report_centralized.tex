\documentclass[11pt]{article}

\usepackage{amsmath}
\usepackage{textcomp}

% Add other packages here %


% Put your group number and names in the author field %
\title{\bf Excercise 4\\ Implementing a centralized agent}
\author{Group \textnumero 29 : Daniel Mock, Dennis Gankin, Donald Duck}


% N.B.: The report should not be longer than 3 pages %


\begin{document}
\maketitle

\section{Solution Representation}

\subsection{Variables}
The idea is to use the provided CSP and decompose a task into a pickup action and delivery action:

nextAction: 
an array of $N_A +N_V$ variables. 
The array contains one variable	for every existing action, and one variable for every existing vehicle:
$nextAction = [nextAction(a1), . . . , nextAction(a_{N_A}
), nextAction(v1), . . ., nextAction(v_{N_V}
)]$
One variable from the nextAction array can take as a value another action,
or the value NULL:
$nextAction(x) \in A \cup \{NULL\}$, where for every task there is a pickup and delivery action:
$A = \{p(t), a(t) \colon t \text{ is a task}\}$.
with the following semantics:
\begin{itemize}
	\item  if $nextAction(a_i) = a_j$ it means that some vehicle will do action $a_j$ immediately after $a_i$;
	\item  if $nextAction(v_k) = a_j$ it means that the vehicle $v_k$ does action $a_j$ (pickup) at first ;
	\item if $nextAction(a_i) = NULL$, the vehicle that did action $a_i$ (delivery) has nothing left to do;
	\item if $nextAction(v_k) = NULL$, the vehicle $v_k$ does not have to any action.
\end{itemize}

One variable from the vehicle array can take as a value the code of the
vehicle that delivers the corresponding task:
$$ vehicle(a) \in V$$

\subsubsection{Implementation}
We implemented this a bit differently:
The datastructure is a map of lists of actions.
Every list corresponds to the actions one vehicle has to perform.
The first element of the list of vehicle $v$ is $nextAction(v)$.
The successor of an action in the list is the result of nextAction.

\subsection{Constraints}
% Describe the constraints in your solution representation %
\begin{enumerate}
	\item $nextAction(a) \neq a$: the action performed after some action $a$ cannot be the same
	action;
	\item All tasks have to be delivered:
	the set of actions in $nextAction$ has to equal $A$ plus $N_V$ times NULL.
	\item The sum of the vehicle's carried tasks weight at any time $i$ may never exceed the vehicle's capacity.
	Be an action either a pickup or a delivery of a task: $a \in \{p(t),d(t)\}$ and be $load(v,i) = \sum_{\substack{
	p(t) \in \{a|vehicle(a)=v\}\\
   	time(p(t)) \leq i  \wedge  time(d(t))>i}} 
	 weight(p(t))$  	
	Then the constraint is: $\forall i,v: load(v,i) \leq capacity(v) $
	\item $nextAction(v) = a \implies vehicle(a) = v$
	\item $nextAction(a) = b \impliedby vehicle(a) = vehicle(b)$
\end{enumerate}
\subsection{Objective function}
% Describe the function that you optimize %
\begin{itemize}
	\item $dist(a,b)$ is the shortest distance between the cities corresponding to the task of $a$ and $b$, i.e.
	if $a$ is a pickup, then the corresponding city is the pickup location of the task of $a$, otherwise the delivery city.
	\item $dist(a, NULL) = 0$, the vehicle stops after the last action.
	\item $dist(v, a)$  is the shortest distance between the home location of the vehicle
	$v$ and the city of $a$;
	\item $dist(v, NULL) = 0$;
	\item $cost(v)$ is the cost per kilometer of vehicle $v$.
\end{itemize}
The total cost of the company is defined as:
\begin{align*}
	C = \sum_{a \in A} dist(a, nextAction(a)) \cdot cost(vehicle(a)) + \sum_{v \in V} dist(v, nextAction(v)) \cdot cost(v)
\end{align*}
\section{Stochastic optimization}

\subsection{Initial solution}
% Describe how you generate the initial solution 
For the initial solution, we assign all actions randomly to the vehicles while being consistent with the constraints. 

\subsection{Generating neighbours}
% Describe how you generate neighbors %
The $chooseNeighbours()$ function provides a set of candidate assignments close to current one and possibly improving it. 
In each iteration, we choose one vehicle at random and use two local operators for finding the neighbours of the current solution.
\begin{description}
	\item $Changing$ $vehicle$: takes the first task from the tasks of one vehicle and give it to another vehicle.

	\item $Changing$ $action$ $order$: takes the according pickup and delivery actions for a vehicle's task and permutes through all possible positions of these actions in the action order. The resulting neighbouring solutions follow the constraints.
\end {description}

\subsection{Stochastic optimization algorithm}
% Describe your stochastic optimization algorithm %


\section{Results}

\subsection{Experiment 1: Model parameters}
% if your model has parameters, perform an experiment and analyze the results for different parameter values %
The model has two parameters: maximum iterations without improvement of the solution and the change probability. The change probability  chooses a new assignment with probability $p$ over the old one if their cost is the same.

\subsubsection{Setting}
% Describe the settings of your experiment: topology, task configuration, number of tasks, number of vehicles, etc. %
% and the parameters you are analyzing %
We use four vehicle in England and seed 12345. We first change one parameter at a time and analyze its influence on the model.

\subsubsection{Observations}
% Describe the experimental results and the conclusions you inferred from these results %
The change probability does not affect the computation a lot. With every value the average final cost is the same around 29900. But the higher the probability that a new solution is picked, the longer the program takes to compute. While for $p=0.1$ the program takes 11 s for $p=0.5$ it takes 12 seconds and for $p=0.9$ it takes 13 seconds. 
Very high maximum iterations without improvement (100 and 1000) show that on average there are 64 iterations with changes to the costs before the improvement stops. Thus a value over 64 will guarantee with a high probability that no possible later changes are cut off. This happens for maximum iterations without improvement set to 10. Even though the program is on average much faster (3 secons compared to 12 seconds with 70) the final cost is 33100 which is much higher than 29900 with more iterations. That is because this way the algorithms stops in a local minimum that it would otherwise have left again after a few iterations. Thus maximum iterations without improvement set to 70 should be a good choice.


\subsection{Experiment 2: Different configurations}
% Run simulations for different configurations of the environment (i.e. different tasks and number of vehicles) %

\subsubsection{Setting}
% Describe the settings of your experiment: topology, task configuration, number of tasks, number of vehicles, etc. %
In this experiment we change the number of tasks and the nubmer of vehicles and analyse how the changes affect the results. We use England with random seed 12345.


\subsubsection{Observations}
% Describe the experimental results and the conclusions you inferred from these results %
The table shows that the amount of tasks is proportional to the number of iterations the program undertakes before there is no improvement anymore. The runtime climbes exponentially tough with more tasks, which is caused by the function that changes the task order by permuting through each task in the vehicle. The costs increase more slowly because there are not so many cities and thus sinergies can be used by the vehicles.
% Reflect on the fairness of the optimal plans. Observe that optimality requires some vehicles to do more work than others. %
Even though the costs per kilometer are all equal we observe that the vehicles carry different amount of tasks. That is because the algorithm optimizez only for the costs. It does not try to have all vehicles be filled with the same amount of tasks. Thus optimal solutions are not necessarily fair in the sense of the amount of carried tasks. 
\begin{tabular}{ c  |c c c }

 tasks & runtime & iterations&cost  \\ 
\hline
 60 & 131s &155 &44300 \\  
 45 & 39s &108 &35741\\
 30 & 12s &64& 29900\\
 15&1s&18&19635
\end{tabular}

% How does the complexity of your algorithm depend on the number of vehicles and various sizes of the task set? %
On top we observe with a fixed amount of tasks: the more vehicles there are, the faster the computations is. That is because this way each vehicle get less tasks and thus the changing task order in one vehicle is much faster. We can see that with less vehicles the runtime grows exponentially.
\begin{tabular}{ c  |c  }

 vehicles & runtime  \\ 
\hline
 4 & 12s  \\  
 3 & 18s \\
 2 & 46s \\
\end{tabular}


\end{document}