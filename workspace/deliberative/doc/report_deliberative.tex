\documentclass[11pt]{article}

\usepackage{amsmath}
\usepackage{textcomp}
\usepackage{hyperref}
% Add other packages here %


% Put your group number and names in the author field %
\title{\bf Exercise 3\\ Implementing a deliberative Agent}
\author{Group \textnumero 29: Daniel A. Mock, Dennis Gankin}


% N.B.: The report should not be longer than 3 pages %


\begin{document}
\maketitle

\section{Model Description}
A state contains the current city of the vehicle of an agent, the carried tasks of an agent and the tasks which are yet to be picked up by the agent.

%\subsection{Intermediate States}
%

\subsection{Goal State}
A state is a goal state if and only if there are no carried tasks and there are no tasks left to be picked up.
This is a necessary and sufficient condition for every task being picked up and delivered.
A state is a intermediate state if and only if it is not a goal state.

\subsection{Actions}
% Describe the possible actions/transitions in your model %
The actions are the three actions given in the logist platform: Either move to a given city, pickup a tasks in the current city or deliver a task in the current city.
Note that we can move to every city, not only the neighboring ones.
The idea behind this is that a sequence of consequent move actions can be fused into one metamove, resulting in a shorter path to a goal state while searching.

\section{Implementation}

\subsection{BFS}
% Details of the BFS implementation %
The BFS is implemented with a queue start contains the nodes that have to be visited and a set of already visited states.
The implementation is as usual, the only interesting thing is how to compute the successor states.

\subsection{A*}
% Details of the A* implementation %
This implementation follows the pseudo code found on the Wikipedia page.
We use a \texttt{Node} to wrap the state and its $f$, $g$ and $h$ values.



\subsection{Common stuff}
Instead of considering move actions to every neighbor city, we consider moves to \emph{relevant cities}:
A city is relevant if 
\begin{itemize}
	\item it is the current city of the vehicle
	\item it is the origin or destination of a task yet to be picked up
	\item it is the destination of a carried task.
\end{itemize}

Finding a goal state is not enough, we have to construct a path from the start state to the goal state.
This is done by remembering from which state we got another state.
This is implemented by a hash map mapping a state to its parent state in the search.
In BFS the parent state corresponds to the first state, that added the state to the search space while expanding.
In A* this parent state may be changed, when a shorter path to this state is found.

Afterwards, we traverse this map from the goal state to the start state and add the corresponding actions to the plan.

\subsection{Heuristic Function}
% Details of the heuristic functions: main idea, optimality, admissibility %
The MST-heuristic:
For a given state, we regard the subgraph of the given network that is induced by the relevant cities of the state.
We compute the minimal spanning tree of this graph.
Since this MST cannot have more weight than a path containing every city of the graph, this heuristic is admissable.

Let $h(s)$ denote the value of state $s$ under the MST heuristic and $d(s,t)$ the cost (travelling distance) to get from state $s$ to state $t$.
To show that A* is optimal, the heuristic has to be monotonic, i.e., we have to show that this inequation holds:
\[ h(s) \leq d(s,t) +h(t) .\]

Since the induced subgraph only changes if the agent delivered or picked up a task and moves to another city in this subgraph, it suffices to show that $MST(G) \leq MST(G - e) + \mathrm{weight}(e)$, where $G$ is a graph, $e$ is an edge from $G$.
Since the set of spanning trees of $G$ is independent, it forms a matroid with the edges as base set and the statement follows from the set exchange property.

We implemented the MST heuristic by using Kruskal's algorithm and the Union-Find data structure to store the connected components.


\section{Results}

\subsection{Experiment 1: BFS and A* Comparison}
% Compare the two algorithms in terms of: optimality, efficiency, limitations %
% Report the number of tasks for which you can build a plan in less than one minute %

\subsubsection{Setting}
% Describe the settings of your experiment: topology, task configuration, etc. %

\subsubsection{Observations}
% Describe the experimental results and the conclusions you inferred from these results %


\subsection{Experiment 2: Multi-agent Experiments}
% Observations in multi-agent experiments %

\subsubsection{Setting}
% Describe the settings of your experiment: topology, task configuration, etc. %

\subsubsection{Observations}
% Describe the experimental results and the conclusions you inferred from these results %

\end{document}