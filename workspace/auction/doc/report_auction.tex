\documentclass[11pt]{article}

\usepackage{amsmath}
\usepackage{textcomp}
\usepackage[top=0.8in, bottom=0.8in, left=0.8in, right=0.8in]{geometry}
% add other packages here

% put your group number and names in the author field
\title{\bf Exercise 5: An Auctioning Agent for the Pickup and Delivery Problem}
\author{Group \textnumero: Student 1, Student 2}

\begin{document}
\maketitle

\section{Bidding strategy}
% describe in details your bidding strategy. Also, focus on answering the following questions:
The general goal of the bidding strategy is to maximize our reward from the bids and at the same time minimize the enemy's reward.
Our bidding strategy combines information from our planner with information about the enemy's won tasks in order to set a bid according to the strategy.
% - do you consider the probability distribution of the tasks in defining your strategy? How do you speculate about the future tasks that might be auctions?
We do not take into account the task distribution.

% - how do you combine all the information from the probability distribution of the tasks, the history and the planner to compute bids?
For the strategy we compute the marginal costs between the original plan a new plan with the auctioned task added. A basic strategy is always bidding the marginal costs. This does not take into account the opponents agent though. In order to make our agent competitive we also try considering the enemy's costs for the auctioned task when bidding. If our costs are lower we want to maximize our reward by bidding as high as we can while still winning the bid. But if the opponent has lower costs than us we want to minimize his reward and thus in this case we should try to bid as low as we can while still losing the bid. Therefore we extend our strategy by bidding the average of our marginal costs and the enemy's assumed marginal costs.
% - how do you use the feedback from the previous auctions to derive information about the other competitors?
In order to compute the opponent's assumed marginal costs we use the feedback from the previous auctions. We store all the tasks won by the enemy and derive the costs from that task set. As we do not know the enemy's configuration as e.g. amount of vehicles, costs per km and so on. we can only assume the costs by computing the marginal costs with our own configuration. Thus the computation is only based on the enemy's acuired tasks but not on its configuration as it is unknown. 




% - how do you combine all the information from the probability distribution of the tasks, the history and the planner to compute bids?

\section{Results}
% in this section, you describe several results from the experiments with your auctioning agent

\subsection{Experiment 1: Comparisons with dummy agents}
% in this experiment you observe how the results depends on the number of tasks auctioned. You compare with some dummy agents and potentially several versions of your agent (with different internal parameter values). 

\subsubsection{Setting}
% you describe how you perform the experiment, the environment and description of the agents you compare with

\subsubsection{Observations}
% you describe the experimental results and the conclusions you inferred from these results

\vdots

\subsection{Experiment n}
% other experiments you would like to present (for example, varying the internal parameter values)

\subsubsection{Setting}

\subsubsection{Observations}

\end{document}
