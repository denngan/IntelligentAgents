\documentclass[11pt]{article}

\usepackage{amsmath}
\usepackage{textcomp}
\usepackage{booktabs}
\usepackage[top=0.8in, bottom=0.8in, left=0.8in, right=0.8in]{geometry}
% Add other packages here %



% Put your group number and names in the author field %
\title{\bf Excercise 1.\\ Implementing a first Application in RePast: A Rabbits Grass Simulation.}
\author{Group \textnumero 29: Dennis Gankin, Daniel A. Mock}

\begin{document}
\maketitle

\section{Implementation}

\subsection{Assumptions}
% Describe the assumptions of your world model and implementation (e.g. is the grass amount bounded in each cell) %
The space is modelled by a torus. Grass is shown green squares and rabbits as white or red squares when the energy drops under 5 points. 
Grass and Rabbits are spreaded randomly. One cell can only contain one grass entity. Rabbits move randomly. Two rabbits cannot be on the same cell at once. A rabbit is born with a randomly chosen amount of energy between 8 and 12 points. Rabbits move randomly. With each step rabbits lose one energy point. When reaching a cell containing grass the rabbit absorbs the grass energy which can be defined by the user. The initial amount of grass and rabbits in the space can also be defined by the user. New grass is created and spreaded randomly every tick. The user can define how many grass entities are spreaded per tick. When an rabbit reaches the energy threshold, also defined by the user a new rabbit is created and added to the space (birth) and the father rabbit falls back to the initial energy level randomly chosen between 8 and 12 points. The space dimensions can also be modified by the user.


\subsection{Implementation Remarks}
% Provide important details about your implementation, such as handling of boundary conditions %
In general valid inputs are recommended. When entering inputs out of the possible bounds, e.g. more rabbits than there is space, the model creates the maximum amount of rabbits but not more and the same with grass. Negative grass energy is possible. Rabbits will lose enrgy when eating grass then. All other negative inputs will be changed to 0. Doubles as input will be read as Integers by Java, so the numbers after the comma will not be taken into account. 

\section{Results}
% In this section, you study and describe how different variables (e.g. birth threshold, grass growth rate etc.) or combinations of variables influence the results. Different experiments with diffrent settings are described below with your observations and analysis

\subsection{Experiment 1: Varying the Energy}

\subsubsection{Setting}
In this experiment, we changed the energy of the grass without changing any other factors.
The BirthThreshold is set to 15 units, the grass growth rate to 15 grass/tick.

\subsubsection{Observations}
If the energy is set to 1, the population dies very slowly without generating any children, since a field of grass can only compensate for the energy consumption of a rabbit per tick.

Otherwise, we see a that the average rabbit population increases with the grass energy.
In the low energy area (2 - 10 energy/grass), we see a linear correlation between grass energy and rabbit population.
This makes sense, since added energy per tick depends on only to parameters: energy per grass and grass growth rate.
If there is twice as much energy in the system, twice the population can be sustained.

If the energy increases more, the population does not increase anymore since the field is ''saturated'' with rabbits (250 rabbits in a 20x20 field).
\begin{table}[]
	\centering
	\caption{Energy = 15, Birth Threshold = 15}
%	\label{my-label}
	\begin{tabular}{@{}llllllll@{}}
		\toprule
		Energy  & 2   & 3   & 4   & 8   & 10  & 40  & 100 \\ \midrule
		Rabbits & 30  & 50  & 60  & 130 & 160 & 250 & 250 \\
		Grass   & 100 & 135 & 100 & 70  & 60  & 50  & 50  \\ \bottomrule
	\end{tabular}
\end{table}

\subsection{Experiment 2: Varying the Growth Rate}

\subsubsection{Setting}
In this we varied the grass growth rate while keeping the birth threshold and energy constant.

\subsubsection{Observations}
In this experiment, the growth rate of the grass was modified.
We used Energy = 2 and Birth Threshold = 20 as other parameters.

The first growth rate where the rabbits did not die instantly, was 8.
Then, we doubled the growth rate which each experiment.
For growth rates smaller than 80, the amount of rabbits doubled as well.
For higher growth rates, the amount does not increase as much, since the space was already filled with grass.
Observe that the grass amount is 290 in a 20\texttimes20 field with a rate of 80.
Thus, a growth rate bigger than 110 does not increase the available grass for the rabbit population.

\begin{table}[]
	\centering
	\caption{Energy = 2, Birth Threshold = 20}
	\label{my-label}
	\begin{tabular}{@{}lllllll@{}}
		\toprule
		Rate    & 10  & 20  & 40  & 80  & 160     & 320     \\ \midrule
		Rabbits & 20  & 40  & 80  & 155 & 180     & 190     \\
		Grass   & 210 & 200 & 230 & 290 & 300-400 & 300-400 \\ \bottomrule
	\end{tabular}
\end{table}

\subsection{Experiment 3: Keeping the Energy\texttimes Rate constant}

\subsubsection{Setting}
We kept the product of grass growth rate and grass energy constant.
The product of the two is 40 and the birth threshold 20.

\subsubsection{Observations}
We observed, that with increasing energy and decreasing growth rate the rabbit population nearly stayed the same.
When the growth got very low, the population fell a bit.
However, the amount of grass fell with the decreasing growth rate.
This makes sense, since the amount of grass only depends on the growth rate and the number of living rabbits.
% Elaborate on the observed results %
\begin{table}[]
	\centering
	\caption{Birth Threshold = 20. The rabbit population stays roughly the same.}
	\label{my-label}
	\begin{tabular}{@{}lllllll@{}}
		\toprule
		Energy  & 4   & 5  & 8     & 10    & 20 & 40 \\ \midrule
		Rate    & 10  & 8  & 5     & 4     & 2  & 1  \\
		Rabbits & 38  & 40 & 30-40 & 28-35 & 28 & 30 \\
		Grass   & 110 & 90 & 60    & 55    & 35 & 30 \\ \bottomrule
	\end{tabular}
\end{table}

\subsection{Experiment 4: Varying the Birth Threshold}
\subsubsection{Setting}
We varied the birth threshold while keeping energy and growth rate constant at 3 and 20 respectively.
\subsubsection{Observations}
For thresholds bigger than 15, the amount of rabbits did not change visibly.
However, the population development seemed to be more smooth the higher the threshold is.
For smaller thresholds, the sustained population increased, since no energy is lost with birth.
\end{document}
